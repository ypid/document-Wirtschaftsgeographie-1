\section{Der primäre Wirtschaftssektor}
\begin{description}
	\item[Der primäre Wirtschaftssektor umfasst:] Die Land- und Forstwirtschaft, Fischerei und Nomadismus
\end{description}
Dies waren die ersten Überlebenstechniken früherer Menschen.

\subsection{Landwirtschaft}
\begin{figurewrapper}
  \begin{tikzpicture}		%% 360*n/100; n=%
    \foreach \start/\end/\middle/\percent/\anchor/\name in {
      0/144/106/40/above/Getreide (Das Hauptnahrungsmittel),
      144/198/171/15/left/Wurzelfrüchte,
      198/252/225/15/left/Gemüse,
      252/298.8/275.4/13/below/Milch,
      298.8/327.6/313.2/8/right/Früchte,
      327.6/360/343.8/9/right/Rest}
	{
    \draw[fill=myblue, thick] (0,0) -- (\end:3cm) arc (\end:\start:3cm)
      node at (\middle:1.8cm) {\percent\,\%};
    \draw (\middle:3cm) -- (\middle:3.5cm) node[\anchor] {\name};
	};
  \end{tikzpicture}
  \captionof{figure}{Kreisdiagramm von 2011 über die Verteilung der Grundnahrungsmittel}
\end{figurewrapper}

\ifDraft{Entwicklung der landwirtschaftlichen Produktion: Siehe Anhang}{}

Das Hauptnahrungsmittel (am meisten verzerrt) ist Getreide.
Die Produktion aller Nahrungsmittel bleibt über 15 Jahre relativ konstant.
Eine Ausnahme bilden die Ölfrüchte.
Sie stieg als chemischer Rohstoff und wegen Treibstoff an (auch Seifen)
\fxwarning{Seifen ???}

\renewcommand{\longtableheader}{\textbf{Mais} & \textbf{Reis}
& \textbf{Weizen} \\}
\newcommand{\Schwelllc}[1]{\textcolor{green}{#1}}
\newcommand{\Entwickllc}[1]{\textcolor{red}{#1}}
\newcommand{\Industrielc}[1]{\textcolor{blue}{#1}}
\newcommand{\Exportll}[1]{\textbf{#1}}
\begin{longtable}{p{3cm}p{3cm}p{3cm}r}
	\longtableheader
	\endfirsthead

	\longtableheader
	\endhead

	\caption{Hauptexporteure von Mais, Reis und Weizen}
	\endlastfoot

	\multicolumn{3}{r}{\longtableendfoot} & \\
	\endfoot

	\Industrielc{\Exportll{USA}}		& \Schwelllc{China}				& \Schwelllc{China}				& \\
	\Schwelllc{China}					& \Schwelllc{Indien}			& \Schwelllc{Indien}
	& ~~~~~\Exportll{Exporteure (Fett)} \\
	\Schwelllc{Brar}					& \Schwelllc{Indones}			& \Industrielc{\Exportll{Europäische Union}}
	& ~~~~~\Schwelllc{Schwellenland} \\
	\Schwelllc{Mexiko}		& \Entwickllc{Bang}			& \Industrielc{\Exportll{USA}}
	& ~~~~~\Entwickllc{Entwicklungsland} \\
	\Schwelllc{\Exportll{Argentinien}}	& \Entwickllc{Vietnam}			& \Schwelllc{Russland}
	& ~~~~~\Industrielc{Industriestaat} \\
										&								& \Schwelllc{Pakistan}			& \\
										&								& \Schwelllc{\Exportll{Kanada}}			& \\
	Subtropen (Kühlgemäßigte Zone)	& Subtropen						& Subtropen (Kühlgemäßigte Zone)&
	 ~~~~~Klimazone\\
\end{longtable}

\subsection{Verwendung der Lebensmittel}
\begin{eqlist}
	\item[Mais:] Futtermittel für Fleisch und Milchproduktion, Rohstoff (Stärke, Energie, Nahrungsmittel)
	\item[Reis:] Nahrungsmittel, Rohstoff
	\item[Weizen:] Nahrungsmittel, Futtermittel $\rightarrow$ Milch, Fleisch, Rohstoff (Stärke, Energie)
\end{eqlist}

\subsection{Voraussetzungen für die Überproduktion in der Landwirtschaft}
\begin{figurewrapper}
  \begin{tikzpicture}[small mindmap, concept color=gray!50, font=\sf, text=white,scale=1.5]
	\node[concept] {}
      child[concept color=\usedcolor, grow=0]{ node[concept,scale=1.5]{Klima- \\ zone}}
      child[concept color=\usedcolor, grow=60]{ node[concept,scale=1.5]{Land}}
      child[concept color=\usedcolor, grow=120]{ node[concept,scale=1.5]{Politik}}
      child[concept color=\usedcolor, grow=180]{ node[concept,scale=1.5]{Produktiv- \\ ität}};
  \end{tikzpicture}
  \captionof{figure}{Mindmap: Voraussetzungen für die Überproduktion in der Landwirtschaft}
\end{figurewrapper}

\subsection{Produktions steigernde Faktoren}
\begin{figurewrapper}
  \begin{tikzpicture}[small mindmap, concept color=gray!50, font=\sf, text=white,scale=1.5]
	\node[concept] {Produktivität}
      child[concept color=\usedcolor, grow=0]{ node[concept,scale=1.5]{Maschinen}}
      child[concept color=\usedcolor, grow=45]{ node[concept,scale=1.5]{Bildung}}
      child[concept color=\usedcolor, grow=90]{ node[concept,scale=1.5]{Infrastruk- \\ tur}}
      child[concept color=\usedcolor, grow=135]{ node[concept,scale=1.5]{Bewässer- \\ ung}}
      child[concept color=\usedcolor, grow=180]{ node[concept,scale=1.5]{Trocken- \\ legung}}
      child[concept color=\usedcolor, grow=225]{ node[concept,scale=1.5]{Speicher}}
      child[concept color=\usedcolor, grow=270]{ node[concept,scale=1.5]{Spritz- \\ mittel}}
      child[concept color=\usedcolor, grow=315]{ node[concept,scale=1.5]{Dünger}};
  \end{tikzpicture}
  \captionof{figure}{Mindmap: Produktions steigernde Faktoren}
\end{figurewrapper}

%\begin{multicols}{2}
\begin{description}
	\item[Maschinen:]~
	\begin{enumerate}[{Pro }1:]
		\item weniger Personalkosten
		\item Qualität und Quantität
		\item Unabhängigkeit
		\item Geschwindigkeit
		\item genauere, effektivere Arbeit
	\end{enumerate}
	\begin{enumerate}[{Kontra }1:]
		\item Umweltbelastung durch:
		\begin{itemize}
			\item Öl Verlust
			\item Reifenabrieb
			\item Schadstoffausstoß (unter anderem \ce{CO2})
			\item Boden Eindrückung/Verdichtung
		\end{itemize}
		\item weniger Personen
		\item teure Anschaffungskosten
		\item Wartungskosten
		\item Fehler durch Automatisierung
		\item Monokultur durch Flächenkonzentration
	\end{enumerate}
	\item[Dünger:]~
	\begin{enumerate}[{Pro }1:]
		\item Qualität-/Quantitätsverbesserung
		\item weniger Spritzmittel durch gesunde Pflanzen
	\end{enumerate}
	\begin{enumerate}[{Kontra }1:]
		\item hohe kosten
		\item Bodenbelastung
	\end{enumerate}
	\item[Spritzmittel:]~
	\begin{enumerate}[{Pro }1:]
		\item weniger Verlust
		\item weniger Schädlinge
		\item Qualität-/Quantitätsverbesserung
	\end{enumerate}
	\begin{enumerate}[{Kontra }1:]
		\item Bodenverpestung
		\item Kosten
		\item Geschmacksveränderung
		\item Umweltschädigung (Gifte im Wasser, Boden)
		\item Qualitätsminderung des Produktes durch Spritzrückstände
	\end{enumerate}
	\item[Trockenlegung:]~
	\begin{enumerate}[{Pro }1:]
		\item mehr Böden
		\item Fruchtbarere Böden
	\end{enumerate}
	\begin{enumerate}[{Kontra }1:]
		\item Kosten (Bau von Staudämmen, Entwässerungsgräben, Wartung)
		\item Bedrohte Artenvielfalt
		\item Zerstörung von Lebensräumen
		\item Veränderung des Grundwasserspiegels
		\item Oberflächenwasser fließt schneller ab
		\item Kosten durch Haltung von Dämmen eventuell Pumpkosten
	\end{enumerate}

%	\columnbreak		%% wenn auf einer Seite
	\ifDraft{}{\newpage}
	\item[Bewässerung:]~
	\begin{enumerate}[{Pro }1:]
		\item Nutzung von Trockenland
		\begin{itemize}
			\item Erhöhung der Entemenge (Quantität)
		\end{itemize}
		\item Wasser zugeben optimieren
		\item Nutzbarmachung von Land (zum Beispiel in Spanien, Ägypten, Israel) durch abpumpen von Grundwasser
	\end{enumerate}
	\begin{enumerate}[{Kontra }1:]
		\item Wasserverbrauch
		\item Absenkung des Grundwasserspiegels
		\item Kosten für Pumpen und ihre Betriebskosten
		\item Übermineralisierung (zum Beispiel Israel -- Salz)
		\item Wegschwämmung von Böden
	\end{enumerate}
	\item[Speicher:]~
	\begin{enumerate}[{Pro }1:]
		\item längere Haltbarkeit
		\item gute Lagerung \entspricht hoher Qualität
	\end{enumerate}
	\begin{enumerate}[{Kontra }1:]
		\item hohe Bau-/Energiekosten
		\item Unterhaltskosten
		\item Landverlust und Zeitverlust
	\end{enumerate}
	\item[Infrastruktur:]~
	\begin{enumerate}[{Pro }1:]
		\item schneller Transport
		\begin{itemize}
			\item Qualität
			\item weniger Transportverlust
		\end{itemize}
		\item Vermarktung der Wahren
		\item Produkt kommt beim Konsumenten an
		\item Personal im Transportgewerbe
	\end{enumerate}
	\begin{enumerate}[{Kontra }1:]
		\item Treibstoffkosten
		\item Versiegelung der Landschaft durch Straßen
		\item Umweltbelastung durch Transport, Abholzung
		\item Personalkosten
		\item In Stand Haltung
	\end{enumerate}
	\item[Bildung:]~
	\begin{enumerate}[{Pro }1:]
		\item Fachwissen ermöglicht Produktivität
		\item gezielter Anbau/Planung der Prozesse
		\item sinnvolles einsetzen von oben angegebenen Hilfsmitteln
		\item Saatgut/Züchtung
		\item Motivation
	\end{enumerate}
	\begin{enumerate}[{Kontra }1:]
		\item Zeit und Geldkosten
	\end{enumerate}
	\item[Saatgut/Züchtung:]~
	\begin{enumerate}[{Pro }1:]
		\item Ertragsmenge erhöhen die Quantität
		\item Qualität
		\item höhere Resistenz
	\end{enumerate}
	\begin{enumerate}[{Kontra }1:]
		\item Jahre langer Zeitaufwand (eine neue Sorte circa 15 Jahre)
		\item Hohe Kosten
	\end{enumerate}
\end{description}
%\end{multicols}

Zentraler Vorteil ist die Steigerung von Qualität und Quantität.
Ein Nachteil aller Faktoren sind die Kosten.

\subsection{Subventionen}
Zuwendungen (finanziell oder materiell) ohne Gegenleistung.

\renewcommand{\longtableheader}{\multicolumn{1}{c}{\textbf{direkt}} & \multicolumn{1}{c}{\textbf{indirekt}}\\}
\begin{longtable}{ll}
	\longtableheader
	\endfirsthead
	\longtableheader
	\endhead
	\caption{Subventionen für Landwirte}
	\endlastfoot
	\multicolumn{2}{r}{\longtableendfoot} \\
	\endfoot

	Flächenstilllegungsprämie & Steuersätze erniedrigt \\
	Darlehen zu geringeren Zinsen & Steuerbefreiung \\
	Zinsen & \\
	Zuschüsse (Fallprämie, Pflanzenprämie) & \\
\end{longtable}

\newpage
\subsection{Daten zu Subventionen}
\renewcommand{\longtableheader}{}
\begin{longtable}{ll}
	\longtableheader
	\endfirsthead
	\longtableheader
	\endhead
	\caption{Daten zu Subventionen}
	\endlastfoot
	\multicolumn{2}{r}{\longtableendfoot} \\
	\endfoot

	EU 2005		& 45\,\% aus EU-Topf in Landwirtschaft\\
	BRD 2010	& \EUR{1,5 Milliarden} zusätzlich\\
	alle Industriestaaten zusammen & 350 Milliarden \$ im Jahr\\
\end{longtable}
