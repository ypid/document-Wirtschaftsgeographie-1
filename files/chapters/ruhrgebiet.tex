%% Nicht im Dokument eingebunden
\renewcommand{\longtableheader}{}
\begin{longtable}{p{0.1\hsize}p{0.5\hsize}p{0.3\hsize}}
	\longtableheader
	\endfirsthead

	\longtableheader
	\endhead

	\caption{Geschichte des Ruhrgebietes}
	\endlastfoot

	\multicolumn{3}{r}{\longtableendfoot}\\
	\endfoot

	14. Jh.	& Im Ruhrtal wird Bergbau (Tagebau) betrieben; abgebaut wird Eisenerz und etwas Kohle,
	zur Verhüttung wird überwiegend Holzkohle eingesetzt, eine ländliche Gegend mit kleinen Dörfern
	& \begin{itemize}\item Handwerker und Landwirte\end{itemize} \\
	18. und 19. Jh. \newline 1. Boomphase & Erfindung der Dampfmaschine $\rightarrow$ betrieb von Pumpen
	(Luft- und Wasser-) Förderkörben ist möglich $\rightarrow$ Kohle wird Untertage abgebaut,
	wertvolle Fett- und Glanzkohle wird abgebaut, Rhein und Ruhr dienen als natürliche Transportwege,
	Krupp gründet das erste Stahlwerk (1849), verarbeitende Betriebe siedeln sich in der Nähe an, ein Ballungsraum entsteht
	& Großer Holzbedarf für Stollenbau, Befeuerung und Bahn
Rodung des Ruhrgebiets
viele Zechen entstehen
viele Menschen werden benötigt
Polen, Tschechen, Slowaken und Italiener werden angeworben
→ Schmelztiegel
Bauern werden zur Arbeitern \\
\end{longtable}