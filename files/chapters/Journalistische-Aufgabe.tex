\section{Kernkraft in der Presse}
Täglich wird der geneigte Leser mit neuen Informationen zu Fukushima Daiichi und im Zuge dessen auch mit den Reaktionen der
einzelnen Länder überhäuft. Aber die Diskussion beziehungsweise die Kritik steht schon über einem halben Jahrhundert im
Raum.

\bigskip
Ich gehe im Folgenden auf die Situation in Japan und speziell auf die Nuklearkatastrophe in dem circa
\SI{9400}{\kilo\meter} entfernten Fukushima ein.

Angefangen hat die Nuklearkatastrophe von Fukushima am \printDate{2011}{03}{11} mit Japans größtem bekannten Erdbeben
(namens \enquote{Tohoku}) der Stärke M\down{W} 9,0, mit vielen Nachbeben, infolgedessen lief
eine Tsunami-Flutwellen auf die
Küste von Japan zu. Unpraktischer weiße stehen direkt an der Pazifikküste mehrere Reaktorblöcke neben Einader, die durch
diesen Tsunami teils schwer beschädigt wurden. Ab diesem Punkt verdrängen die Nachrichten über Fukushima Daiichi etwas die
anderen folgen des Erdbebens. Die direkte Folge war eine Überschwemmung der Turbinengebäude und der Notstromgeneratoren.
Die Blöcke eins bis drei schalteten sich aufgrund des Bebens zwar selbst ab, aber aufgrund dessen das fast alle Notstromgeneratoren
nicht mehr funktionsfähig waren lässt sich die nicht zu unterschätzende Nachzerfallswärme der Brennstäbe nicht mehr
ausreichend abführen, was im schlimmsten Fall zu einer Kernschmelze führen kann.

Auch die Batterien, die bereitstehen, um eine Kernschmelze zu verhindern, fallen durch die Überschwemmung aus. So beginnt
ein langer Kampf, um die Katastrophe noch zu begrenzen. Es wird zum Beispiel Meerwasser zur Kühlung in den Reaktorkern und
die Abklingbecken eingeleitet. Wodurch aber eine Salzkruste entsteht.
Später treten noch massive Probleme mit hoch radioaktiv verstrahltem Wasser, was ins Meer
zurückfliest. Am \printDate{2011}{04}{12} erhöhte die japanische Atomaufsicht
die Gefahrenstufe von fünf auf sieben, was das Maximum dieser Skala darstellt und somit steht die
Nuklearkatastrophe in Fukushima auf einer Ebene mit der
Katastrophe in Tschernobyl (1986).
Zudem wird vermutet, dass es in drei Reaktorblöcken wahrscheinlich zu einer teilweisen Kernschmelze kam.
Eine der wenigen Möglichkeiten, die einem bei so einem GAU noch bleiben, ist die Erbauung eines  Sarkophags um die
Reaktoren, wie dies auch bei Tschernobyl getan wurde.

\bigskip
Auswirkungen der Ereignisse sind zum Beispiel die von Angela Merkel verkündeten Abschaltungen der ältesten Kernreaktoren in
Deutschland und eine Sicherheitsüberprüfung der anderen Kernreaktoren. Und die Erhöhung der Grenzwerte für radioaktiv
belastete Lebensmittel durch die EU, was sich mir nicht erschließt.

Aber worauf ich noch warte/hoffe ist ein wirkliches Umdenken, sich fürs Erste komplett von dieser momentan offenbar
unkontrollierbaren und risikoreichen Energieform Kernenergie für Waffen oder die Energieerzeugung in Reaktoren zu trennen.

Es ist bedauerlich das erst durch solche Zwischenfälle ein Denkprozess angestoßen wird und
etwas passiert, wie das vorher erwähnte Abschalten einiger alter Kernkraftwerke in Deutschland.
Was mir aber mehr wie eine geforderte Reaktion erscheint, die den Eindruck erwecken soll, dass etwas geschieht.

\bigskip
Punkte, die ich auf der Strecke gelassen habe, aber hier noch kurz erwähnen möchte sind unter anderem die nicht geklärte
Frage
der Endlagerung oder die Möglichkeit von Terroristen ein zweites Tschernobyl durch die Auslösung einer Kernschmelze in
einem Castorbehälter.

\begin{description}
	\item[Weitergehende Links:]~
	\begin{itemize}
		\item \url{http://de.wikipedia.org/wiki/Kernkraftwerk_Fukushima_Daiichi}
		\item \url{http://de.wikipedia.org/wiki/Chronik_der_Nuklearkatastrophe_von_Fukushima}
		\item \url{http://alternativlos.org/14/}
	\end{itemize}
\end{description}
