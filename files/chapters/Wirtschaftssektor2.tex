\section{Sekundärer Wirtschaftssektor}
\subsection{Stahlindustrie}

\begin{description}
	\item[Bergbau:]~
	\begin{itemize}
		\item Über- oder Untertage wird der Energieträger Stein- und Braunkohle abgebaut
		\item Über- oder Untertage wird der Rohstoff Eisenerz abgebaut
		\item effektiver Untertagebau wurde erst durch die Erfindung der Dampfmaschine möglich
			-- Transport -- Be- und Entlüftung -- Abpumpen von Grundwasser
	\end{itemize}
	\item[Kokerei:]~
	\begin{itemize}
		\item Kohle wird unter Luftabschluss erhitzt -- verschwehlt
		\item unerwünschte Stoffe werden als Gas ausgetrieben
		\item der Brennwert der Kohle wird erhöht, es entsteht Koks
		\item zum Schluss wird der Koks mit Wasser gelöscht
	\end{itemize}
	\item[Hochofen:]~
	\begin{itemize}
		\item Eisenerz, Kalk und Koks werden in mehreren Lagen übereinander geschichtet und der Koks entzündet
		\item ein Teil des Koks verbrennt und liefert die Temperatur für die Reaktion und den Schmelzprozess
		\item der andere Teil des Koks wird reduziert und liefert die Elektronen des Eisenerz \ce{Fe^{2+/3+}}
		\begin{itemize}
			\item \ce{Fe^0}, also Roheisen
		\end{itemize}
	\end{itemize}
	\item[Stahlwerk:]~
	\begin{itemize}
		\item Nach dem Thomson- oder Simensverfahren wird das Roheisen zu Stahl veredelt
		\item Roheisen wird aufgeschmolzen
		\item Unerwünschte Stoffe, wie Schwefel unter anderem werden entfernt
		\item Stoffe, die zu der gewünschten Stahleigenschaft führen werden,
			zugegeben (Molybdän, Chrom, Nickel, Stickstoff unter anderem)
		\item Anschließend wird der flüssige Stahl in Blöcke, Brammen oder Knüppel gegossen
	\end{itemize}
	\item[Walzwerk:]~
	\begin{itemize}
		\item Die angelieferten Brammen werden auf die gewünschte Temperatur gebracht
		\item Mehrstufige Walzen bringen die Brammen in die gewünschte Form: Bleche, Profile, Drähte, Rohre usw.
	\end{itemize}
\end{description}

\ifDraft{Abbildung zur Stahlherstellung: Siehe Anhang}{}

\subsection{Wozu wird Stahl gebraucht?}
\begin{multicols}{2}
\begin{itemize}
	\item Maschinenbau und Werkzeuge
	\item Bauindustrie
	\item Transport (Auto, Schiffe, Züge,\newline Flugzeuge)
	\item Schienennetz
	\item Waffen und Munition
	\item Stromtransport (Masten und Kabel)
	\item Rohre
	\item Kommunikation (Antenne)
	\item Produkte (Dosen, Schlüssel, Gehäuse...)
\end{itemize}

Stahl ist:
\begin{itemize}
	\item selbst das Produkt
	\item Voraussetzung für Produktion
	\item Voraussetzung für Transport
	\begin{itemize}
		\item elementare Säule der Industriegesellschaft
	\end{itemize}
\end{itemize}
\end{multicols}

\ifDraft{Geschichte des Ruhrgebietes: Siehe Anhang}{}
